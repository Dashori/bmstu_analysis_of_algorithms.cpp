\chapter*{Заключение}
\addcontentsline{toc}{chapter}{Заключение}

В ходе выполнения лабораторной работы были решены все задачи:
\begin{itemize}
	\item изучены алгоритмы нахождения расстояния Левенштейна и Дамерау -- Левенштейна;
	\item применены методы динамического программирования для реализации алгоритмов;
	\item на основе полученных в ходе экспериментов данных были сделаны выводы по поводу эффективности всех реализованных алгоритмов;
	\item был подготовлен отчет по лабораторной работе.
\end{itemize}

Результат замерных экспериментов реализованных алгоритмов показал, что самым быстрым алгоритмом является алгоритм поиска расстояния Левенштейна. Начиная с длины слова 10 он превосходит итеративный и рекурсивный с кешом алгоритм поиска расстояния Дамерау -- Левенштейна в 1.5 и 2 раза соответственно. Рекурсивный алгоритм с заполнением матрицы превосходит простой рекурсивный.

При сравнении используемой памяти реализаций итеративных алгоритмов проигрывают рекурсивному без кеширования. В свою очередь реализация итеративного алгоритма нахождения расстояния Левенштейна занимает меньше памяти, чем  реализация итеративного алгоритма нахождения расстояния Дамерау -- Левенштейна. Это связано с добавлением редакторской операции транспозиции. При этом итеративные и рекурсивный с кэшированием реализации алгоритмов сопоставимы по объёму занимаемой памяти.
