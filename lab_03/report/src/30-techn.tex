\chapter{Технологическая часть}

\section{Требования к ПО}

К программе предъявляется ряд требований:
\begin{itemize}
	\item входными данными являются размер массива и непосредственно элементы массива;
	\item на выходе -- отсортированный массив.
\end{itemize}

\section{Средства реализации}

В качестве языка программирования для реализации лабораторной работы был выбран C++ -- компилируемый, статически типизированный язык программирования общего назначения \cite{cpp}. 

Данный выбор обусловлен поддержкой языком парадигмы объектно -- ориентированного программирования и наличием методов для замера процессорного времени.

Время работы реализованных алгоритмов было замерено с помощью библиотеки chrono\cite{chrono}.

\section{Реализация алгоритмов}

В листингe \ref{lst:merge} приведена реализация алгоритма сортировки слиянием.
\lstinputlisting[caption={Реализация алгоритма сортировки слиянием}, label={lst:merge}]{code/merge.cpp}

В листингe \ref{lst:counting} приведена реализация алгоритма сортировки подсчетом.
\lstinputlisting[caption={Реализация алгоритма сортировки подсчетом}, label={lst:counting}]{code/counting.cpp}

В листингe \ref{lst:bitonic} приведена реализация алгоритма битонной сортировки.
\lstinputlisting[caption={Реализация алгоритма битонной сортировки}, label={lst:bitonic}]{code/bitonic.cpp}
