\chapter*{Введение}
\addcontentsline{toc}{chapter}{Введение}

Одной из важнейших процедур обработки структурированной информации является сортировка \cite{Knut}. Сортировкой называют процесс перегруппировки заданной последовательности (кортежа) объектов в некотором определенном порядке. Определенный порядок (например, упорядочение в алфавитном порядке, по возрастанию или убыванию количественных характеристик, по классам, типам и.т.п.) в последовательности объектов необходимо для удобства работы с этим объектом. В частности, одной из целей сортировки является облегчение последующего поиска элементов в отсортированном множестве. 

Любой алгоритм сортировки можно разбить на три основные части:
\begin{itemize}
	\item сравнение элементов для определения их упорядоченности;
	\item перестановка элементов;
	\item сортирующий алгоритм, который осуществляет сравнение и перестановку элементов до тех пор, пока все элементы не будут упорядочены.
\end{itemize}

Важнейшей характеристикой любого алгоритма сортировки является скорость его работы, которая определяется функциональной зависимостью среднего времени сортировки последовательностей элементов данных, заданной длины, от этой длины. Время сортировки будет пропорционально количеству сравнений и перестановки элементов данных в процессе их сортировки.

Цель лабораторной работы -- изучить и исследовать трудоемкость алгоритмов сортировки.

Задачи лабораторной работы:
\begin{itemize}
	\item изучить и реализовать 3 алгоритма сортировки: слиянием, подсчетом, битонная;
	\item выбрать инструменты для процессорного времени выполнения реализаций алгоритмов;
	\item провести анализ затрат работы программы по времени;
	\item подготовить отчет по лабораторной работе.
\end{itemize}
