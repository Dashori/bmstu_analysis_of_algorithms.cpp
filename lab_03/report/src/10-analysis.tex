\chapter{Аналитическая часть}

\section{Сортировка слиянием}

Алгоритм сортировки слиянием в большой степени соответствует парадигме метода разбиения~\cite{cormen}.
Сортировка слиянием применяется для структур данных, доступ к элементам которых можно получать только последовательно, например, списки или массивы.

Приведем алгоритм сортировки для массива:
\begin{enumerate}
	\item сортируемый массив разбивается на две части примерно одинакового размера;
	\item каждая из получившихся частей сортируется отдельно тем же самым алгоритмом; 
	\item два упорядоченных массива половинного размера соединяются в один.
\end{enumerate}
Рекурсивное разбиение задачи на меньшие происходит до тех пор, пока размер массива не достигнет единицы. Для упрощения реализации алгоритма можно заранее создать временный массив и передать его в качестве аргумента функции. Сортировка является устойчивой.

\section{Сортировка подсчетом}

Сортировка подсчетом применяется для структур данных, доступ к элементам которых только последовательно, например, списки или массивы. Используется диапазон чисел сортируемого массива (списка) для подсчёта совпадающих элементов~\cite{knut}. 

Приведем алгоритм сортировки для массива:
\begin{enumerate}
	\item в сортируемом массиве находится максимальный и минимальный элемент;
	\item создается дополнительный массив, размер которого -- разница между максимальным и минимальным элементом;
	\item считается колличество каждого числа сортируемого массива и результат записывается в дополнительный массив;
	\item с помощью дополнительного массива формируется отсортированный массив.
\end{enumerate}
Если в массиве используются только натуральные числа, то минимальный элемент находить не требуется, он будет равен 0. Применение сортировки подсчётом целесообразно когда сортируемые числа имеют диапазон возможных значений, который достаточно мал по сравнению с сортируемым множеством, например, миллион натуральных чисел меньших 1000. Сортировка является устойчивой.


\section{Битонная сортировка}

Алгоритм применяется для массивов, размер которых степень двойки, так как он рекурсивно делит массив пополам. Из-за этого может понадобиться добавлять фиктивные элементы в сортируемый массив, что не влияет на асимптотику. Алгоритм основан на сортировке битонных последовательностей. Последовательность называется битонической, если она монотонно возрастает, а затем монотонно убывает, или если путем циклического сдвига ее можно привести к такому виду~\cite{cormen}.

Приведем алгоритм сортировки для массива:
\begin{enumerate}
	\item сортируемый массив преобразуется в битонную последовательность;
	\item сортируемый массив разбивается на две части одинакового размера и также преобразовывается в битонную последовательность;
	\item два упорядоченных массива половинного размера соединяются в один.
\end{enumerate}
Параллельный алгоритм сортировки применяется для создания сортировочных сетей. Сортировка является устойчивой.