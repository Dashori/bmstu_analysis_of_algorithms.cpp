\chapter*{Введение}
\addcontentsline{toc}{chapter}{Введение}

Расстояние Левенштейна (редакционное расстояние) -- метрика, измеряющая разность между двумя последовательностями символов. Она определяется как минимальное количество односимвольных операций (а именно вставки, удаления, замены), необходимых для превращения одной последовательности символов в другую. Впервые задачу нахождения редакционного расстояния поставил в 1965 году советский математик Владимир Левенштейн при изучении последовательностей, состоящих из  0 и 1 \cite{bib1}.

Расстояние Дамерау -- Левенштейна (названо в честь учёных Фредерика Дамерау и Владимира Левенштейна) является модификацией расстояния Левенштейна: к операциям вставки, удаления и замены символов, определённых в расстоянии Левенштейна добавлена операция транспозиции (перестановки) символов.

Расстояние Левенштейна и похожие расстояния активно применяются: 
\begin{enumerate}[label={\arabic*)}]
	\item для исправления ошибок в слове (в поисковых системах, базах данных, при вводе текста, при автоматическом распознавании отсканированного текста или речи);
	\item для сравнения текстовых файлов утилитой \code{diff} и ей подобными (здесь роль «символов» играют строки, а роль «строк» — файлы);
	\item в биоинформатике для сравнения генов, хромосом и белков.
\end{enumerate}


Задачи лабораторной работы:

\begin{itemize}
	\item изучение алгоритмов нахождения расстояния Левенштейна и Дамерау -- Левенштейна;
	\item применение методов динамического программирования для реализации алгоритмов поиска расстояния Левештейна и Дамерау -- Левенштейна;
	\item сравнительный анализ алгоритмов на основе экспериментальных данных;
	\item подготовка отчета по лабораторной работе.
\end{itemize}
