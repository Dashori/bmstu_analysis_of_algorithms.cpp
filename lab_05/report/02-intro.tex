\section*{ВВЕДЕНИЕ}
\addcontentsline{toc}{section}{ВВЕДЕНИЕ}

Целью данной лабораторной работы является изучение конвейерных вычислений.

При обработке данных могут возникать ситуации, когда один набор данных необходимо обработать последовательно несколькими алгоритмами. В таком случае удобно использовать конвейерную обработку данных, что позволяет на каждой следующей <<линии>> конвейера использовать данные, полученные с предыдущего этапа.
Отдельно стоит упомянуть асинхронные конвейерные вычисления. Отличие от линейных состоит в том, что при таком подходе линии работают с меньшим временем простоя, так как могут обрабатывать задачи независимо от других линий.

Задачи лабораторной работы:
\begin{itemize}[label*=---]
	\item рассмотреть и изучить асинхронную конвейерную обработку данных;
	\item реализовать систему конвейерных вычислений с тремя линиями;
	\item провести сравнительный анализ по времени последовательной и конвейерной реализаций;
	\item подготовить отчет по лабораторной работе.
\end{itemize}
