\section{Аналитическая часть}

\subsection{Предметная область лабораторной работы}
В качестве алгоритма, реализованного для распределения на конвейере, было выбрано абстрактное оформление банковской карты, состоящее из трех этапов:
\begin{enumerate}[label*=---]
	\item генерация данных о владельце карты (ФИО,  пол);
	\item генерация данных о карте (платежная система, номер карты в зависимости от платежной системы, CVV код);
	\item запись сгенерированных данных в CSV-файл.
\end{enumerate}

Каждый из описанных выше этапов будет выполняться на отдельной ленте.

\subsection{Последовательный алгоритм}

При последовательном алгоритме создание нескольких банковских карт не может происходить одновременно. То есть,  пока новая банковская карта не пройдет все три линии, то новая не начнет обрабатываться.

\subsection{Параллельный алгоритм}

В случае параллельного для каждой ленты создается отдельный поток. Извлечение и добавление банковских карт осуществляется потоками. Так, в параллельном алгоритме:
\begin{enumerate}[label*=---]
	\item первый поток извлекает карту из первой очереди, она обрабатывается на первой ленте, и поток добавляет ее во вторую очередь;
	\item второй поток извлекает карту из второй очереди, она обрабатывается на второй ленте, и поток добавляет ее в третью очередь;
	\item третий поток извлекает карту из второй очереди, и она обрабатывается на третьей ленте;
	\item каждый поток по завершении вновь извлекает карту из своей очереди, и цикл обработки повторяется в параллельном режиме.
\end{enumerate}
