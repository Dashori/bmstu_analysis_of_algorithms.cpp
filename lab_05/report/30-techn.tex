\section{Технологическая часть}

\subsection{Требования к программному обеспечению}

К программе предъявляется ряд требований:
\begin{itemize}[label*=---]
	\item на вход подается количество карт, которое надо создать -- целое положительное число;
	\item на выходе -- время, затраченное на обработку карт;
	\item в процессе обработки задач необходимо фиксировать время прихода и ухода карты с линии.
\end{itemize}

\subsection{Средства реализации}

В качестве языка программирования для реализации лабораторной работы был выбран C++ -- компилируемый, статически типизированный язык программирования общего назначения \cite{cpp}. 

Данный выбор обусловлен поддержкой языком парадигмы объектно -- ориентированного программирования и наличием методов для замера процессорного времени.

Время работы реализованных алгоритмов было замерено с помощью библиотеки chrono \cite{chrono}.

\subsection{Реализация алгоритмов}

В листингe \ref{lst:merge} приведена реализация последовательного алгоритма оформления банковской карты.
\begin{code}
	\captionof{listing}{Реализация последовательного алгоритма оформления банковской карты}
	\label{lst:merge}
	\inputminted
	[
	frame=single,
	framerule=0.5pt,
	framesep=20pt,
	fontsize=\small,
	tabsize=4,
	linenos,
	numbersep=5pt,
	xleftmargin=10pt,
	]
	{text}
	{code/linear.cpp}
\end{code}

В листингe \ref{lst:counting} приведена реализация параллельного алгоритма оформления банковской карты.
\pagebreak
\begin{code}
	\captionof{listing}{Реализация  параллельного алгоритма оформления банковской карты}
	\label{lst:counting}
	\inputminted
	[
	frame=single,
	framerule=0.5pt,
	framesep=20pt,
	fontsize=\small,
	tabsize=4,
	linenos,
	numbersep=5pt,
	xleftmargin=10pt,
	]
	{text}
	{code/parallel.cpp}
\end{code}

