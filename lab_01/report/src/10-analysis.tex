\chapter{Аналитическая часть}

Расстояние Дамерау -- Левенштейна между двумя строками -- количество редакторских операций необходимых для преобразования одной строки в другую.

Редакторские операции:
\begin{itemize}
	\item вставка (англ. insert);
	\item удаление (англ. delete);
	\item замена (англ. replace).
\end{itemize}

Для расстояния Дамерау -- Левенштейна вводится редакторская операция транспозиция (англ. transposition).

В общем случае:
\begin{itemize}
	\item $w(a,b) = 1$ -- цена замены символа $a$ на символ $b$, $a \neq b$;
	\item $w(\lambda,b) = 1$ -- цена вставки символа $b$;
	\item $w(a,\lambda) = 1$ -- цена удаления символа $a$;
	\item $w(ab, ba) == 1$  -- цена транспозиции двух соседних символов;
	\item $w(a,a) = 0$ -- цена совпадения.
\end{itemize}

\newpage
\section{Нерекурсивный алгоритм поиска расстояния Левенштейна}
Пусть $S_{1}$ и $S_{2}$ -- две строки длиной len($S_{1}$) и len($S_{2}$) соответственно, над некоторым конечным алфавитом.

Расстояние Левенштейна может быть найдено по формуле \ref{eq:d}, которая задана как:
\begin{equation}
	\label{eq:d}
	D(i, j) = \begin{cases}
		\max(i, j), &\text{если }\min(i, j) = 0,\\
		\min \lbrace \\
		\qquad D(i, j-1) + 1,\\
		\qquad D(i-1, j) + 1,\\
		\qquad D(i-1, j-1) + m(S_{1}[i], S_{2}[j]), &\text{ иначе}\\
		\rbrace
	\end{cases},
\end{equation}
где функция $m$ определена как:
\begin{equation}
	\label{eq:m}
	m(S_{1}[i], S_{2}[j]) = \begin{cases}
		0, &\text{если $S_{1}[i] = S_{2}[j]$,}\\
		1, &\text{иначе}
	\end{cases}.
\end{equation}

Для оптимизации нахождения расстояния используется матрица промежуточных значений. Ее размерность $(len(S_{1})+ 1) \times (len(S_{2}) + 1)$.

Значение в ячейке $matrix[i, j]$ равно значению $D(S_{1}[1...i], S_{2}[1...j])$. Первая строка и первый столбец тривиальны и совпадают с наибольшим значением индекса $i$ или $j$ ячейки. 

Вся матрица (кроме первого столбца и первой строки) заполняется в соответствии с формулой \ref{eq:mat}:
\begin{equation}
	\label{eq:mat}
	matrix[i][j] = min \begin{cases}
		\qquad matrix[i-1][j] + 1,\\
		\qquad matrix[i][j-1] + 1,\\
		\qquad matrix[i-1][j-1] + m(S_{1}[i], S_{2}[j]),\\
		\qquad \left[ \begin{array}{cc}matrix[i-2][j-2] + 1, &\text{если }i,j > 1;\\
			\qquad &\text{}S_{1}[i] = S_{2}[j-1];\\
			\qquad &\text{}S_{1}[i-1] = S_{2}[j]\end{array}\right.\\	
	\end{cases}.
\end{equation}

В результате расстоянием Левенштейна будет ячейка матрицы с индексами $i = len(S_{1}$) и $j = len(S_{2})$.

\section{Нерекурсивный алгоритм поиска расстояния Дамерау -- Левенштейна}

Расстояние Дамерау -- Левенштейна может быть найдено по формуле \ref{eq:d1}, которая задана как:
\begin{equation}
	\label{eq:d1}
	D(i, j) = \begin{cases}
		\max(i, j), &\text{если }\min(i, j) = 0,\\
		\min \lbrace \\
		\qquad D(i, j-1) + 1,\\
		\qquad D(i-1, j) + 1,\\
		\qquad D(i-1, j-1) + m(S_{1}[i], S_{2}[j]), &\text{ иначе}\\
		\qquad \left[ \begin{array}{cc}D(i-2, j-2) + 1, &\text{если }i,j > 1;\\
			\qquad &\text{}S_{1}[i] = S_{2}[j-1];\\
			\qquad &\text{}S_{1}	[i-1] = S_{2}[j]  \\
			\qquad \\
			\qquad \infty, & \text{иначе}\end{array}\right.\\
		\rbrace
	\end{cases},
\end{equation}

Формула выводится по тем же соображениям, что и формула \ref{eq:d}, но с добавлением редакторской операции транспозиции.

В результате расстоянием Дамерау -- Левенштейна будет ячейка матрицы с индексами $i = len(S_{1}$) и $j = len(S_{2})$.

\section{Рекурсивный алгоритм поиска расстояния Дамерау -- Левенштейна}

Рекурсивный алгоритм реализует формулу \ref{eq:d1}.
Функция $D$ составлена из следующих соображений:
\begin{enumerate}[label={\arabic*)}]
	\item для перевода из пустой строки в пустую требуется ноль операций;
	\item для перевода из пустой строки в строку $S_{1}$ требуется $|S_{1}|$ операций;
	\item для перевода из строки $S_{1}$ в пустую требуется $|S_{1}|$ операций;
\end{enumerate}

Для перевода из строки $S_{1}$ в строку $S_{2}$ требуется выполнить последовательно некоторое количество операций (удаление, вставка, замена, транспозиция) в некоторой последовательности. 

Полагая, что $S_{1}', S_{2}'$  и $S_{1}'', S_{2}''$ — строки $S_{1}$ и $S_{2}$ без последнего символа и без двух последних символов соответственно, цена преобразования из строки $S_{1}$ в строку $S_{2}$ может быть выражена как:
\begin{enumerate}[label={\arabic*)}]
	\item сумма цены преобразования строки $S_{1}'$ в $S_{2}$ и цены проведения операции удаления, которая необходима для преобразования $S_{1}'$ в $S_{1}$;
	\item сумма цены преобразования строки $S_{1}$ в $S_{2}'$  и цены проведения операции вставки, которая необходима для преобразования $S_{2}'$ в $S_{2}$;
	\item сумма цены преобразования из $S_{1}'$ в $S_{2}'$ и операции замены, предполагая, что $S_{1}$ и $S_{2}$ оканчиваются на разные символы;
	\item цена преобразования из $S_{1}'$ в $S_{2}'$, предполагая, что $S_{1}$ и $S_{2}$ оканчиваются на один и тот же символ.
	\item сумма цены преобразования $S_{1}''$ в $S_{2}''$, при условии, что два последних символа $S_{1}$ равны двум последним символам $S_{2}$ после транспозиции;
\end{enumerate}

Минимальной ценой преобразования будет минимальное значение из приведенных вариантов.

\section{Рекурсивный с кешированием алгоритм поиска расстояния Дамерау -- Левенштейна}

Рекурсивный алгоритм заполнения можно оптимизировать по времени выполнения с использованием матричного алгоритма. Суть данного метода заключается в заполнении матрицы при выполнении рекурсии. В случае, если рекурсивный алгоритм выполняет прогон для данных, которые еще не были обработаны, результат нахождения расстояния заносится в матрицу. В случае, если обработанные ранее данные встречаются снова, то есть ячейка матрицы уже заполнена, для них расстояние не находится и алгоритм переходит к следующему шагу.
