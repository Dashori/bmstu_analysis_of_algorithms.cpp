\chapter*{Заключение}
\addcontentsline{toc}{chapter}{Заключение}

Цель работы достигнута: изучен метод динамического программирования на примере редакционных расстояний.

В ходе выполнения лабораторной работы были решены все задачи:
\begin{itemize}
	\item изучены алгоритмы нахождения расстояния Левенштейна и Дамерау -- Левенштейна;
	\item применены методы динамического программирования для реализации алгоритмов;
	\item на основе полученных в ходе экспериментов данных были сделаны выводы по поводу эффективности всех реализованных алгоритмов;
	\item был подготовлен отчет по лабораторной работе.
\end{itemize}

Эксперименты показали, что наиболее затратный по времени рекурсивный алгоритм поиска расстояния Дамерау -- Левенштейна без кеша, а наименее затратны итеративные алгоритмы. Менее затратными по памяти являются реализации итеративных алгоритмов. Самый затратный по памяти является оеализация рекурсивного алгоритма поиска расстояния Дамерау -- Левенштейна с кешированием. 