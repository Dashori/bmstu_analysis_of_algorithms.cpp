\chapter{Технологическая часть}

В данном разделе приведены требования к программному обеспечению, средства реализации и листинги кода.

\section{Требования к ПО}

К программе предъявляется ряд требований:
\begin{itemize}
	\item на вход подаются две строки на русском или английском языке в любом регистре;
	\item на выходе -- искомое расстояние для трех методов, матрица расстояний для итеративного метода и время выполнения в тиках.
\end{itemize}

\section{Средства реализации}

В качестве языка программирования для реализации лабораторной работы был выбран C++ -- компилируемый, статически типизированный язык программирования общего назначения \cite{cpp}. 

Данный выбор обусловлен поддержкой языком парадигмы объектно -- ориентированного программирования и наличием методов для замера процессорного времени.

Время выполнения реализации алгоритмов было замерено с помощью ассемблерной вставки и измеряется в тиках.

\section{Сведения о модулях программы}
Программа состоит из трех программных модулей:
\begin{enumerate}[label={\arabic*)}]
	\item main.cpp -- главный модуль программы, содержащий функцию main, с которой начинается выполнение программы;
	\item algorithm.cpp, algorithm.h -- модуль с реализацией алгоритмов;
	\item timer.cpp, timer.h -- модуль для замера времени.
\end{enumerate}

\section{Реализация алгоритмов}

В листингe \ref{lst:algorithms1} приведена реализация итеративного алгоритма нахождения расстояния Дамерау -- Левенштейна, а также вспомогательные функции.

%\lstinputlisting[caption={Итеративный алгоритм}, label={lst:algorithms1}]{iterativeDM.cpp}
\begin{lstinputlisting}[
	caption={Итеративный алгоритм},
	label={lst:algorithms1}
	]{iterativeDM.cpp}
\end{lstinputlisting}

В листингe \ref{lst:algorithms2} приведена реализация рекурсивного алгоритма нахождения расстояния Дамерау -- Левенштейна, а также вспомогательные функции.

\lstinputlisting[caption={Рекурсивный алгоритм без кеша}, label={lst:algorithms2}]{recursiveDM.cpp}

В листингe \ref{lst:algorithms3} приведена реализация рекурсивного алгоритма нахождения расстояния Дамерау -- Левенштейна с кешированием, а также вспомогательные функции.
\pagebreak
\lstinputlisting[caption={Рекурсивный алгоритм с кешированием}, label={lst:algorithms3}]{recursiveCacheDM.cpp}