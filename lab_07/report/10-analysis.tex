\section{Аналитический раздел}

\subsection{Объект и его признак}
Словарь \cite{slov} -- абстрактный тип данных, который позволяет хранить данные в виде пар <<ключ-значение>>, с возможностью использования операции добавления пары, а также поиска и удаления пары по ключу. В паре $(k, v)$ значение $v$ называется значением, ассоциированным с ключом $ k $.
Поиск -- основная задача при использовании словаря. Данную задачу можно решить различными способами.
В данной лабораторной работе словарь используется для описания объекта <<количество минут -- длительной шахматной партии>> со следующими параметрами: ключ -- терм (словесное
описание признака), значение -- массив числовых значений признака. Доступные числовые значения признака: от 1 минуты до 1000 минут.

Доступные термы:
\begin{itemize}
	\item[1)] очень мало;
	\item[2)] мало;
	\item[3)] не очень мало;
	\item[4)] средне;
	\item[5)] не очень много;
	\item[6)] много;
	\item[7)] очень много.
\end{itemize}


\subsection{Анкетирование респондентов}
В таблице \ref{tbl:fio} представлены данные проведенного анкетирования по оценке респондентами количества минут длительности шахматной партии.

\begin{table}[h!]
	\captionsetup{justification=raggedright, singlelinecheck=off, margin*=33pt}
	\caption{\label{tbl:fio} Данные анкетирования}
	\begin{center}
		\begin{tabular}{|c|c|c|c|c|c|c|c|}
			\hline
			& \multicolumn{7}{c|}{Термы в порядке указанном в прошлой главе} \\
			\cline{2-8} 
			\raisebox{1.5ex}[0cm][0cm]{ФИО респондента}
			& 1 & 2 & 3 & 4 & 5 & 6 & 7 \\
			\hline
			Егорова & 1 & 2 & 3 & 4 & 5 & 6 - 7 & 8 - 12 \\
			\hline
			Шабанова   & 1 - 2 & 3 - 4 & 5 & 5 - 6 & 7 & 8 - 11 & 12 \\
			\hline
			Татаринова & 1 & 2 - 4 & 4 - 5 & 5 & 6 & 7 - 10 & 11 - 12 \\
			\hline
			Фролова & 1 & 2 - 4 & 4 - 5 & 5 & 6 & 7 - 10 & 11 - 12 \\
			\hline
		\end{tabular}
	\end{center}
\end{table}

\pagebreak
На основании проведенного исследования можно сформулировать несколько типовых вопросов, на которые сможет ответить программное обеспечение:
\begin{itemize}
	\item[---] <<Сколько минут много для проведения шахматной партии?>>
	\item[---] <<Достаточно ли  60 минут для проведения шахматной партии?>>
	\item[---] <<Для проведения шахматной партии 2 часа это много?>>
\end{itemize}
