\section*{ВВЕДЕНИЕ}
\addcontentsline{toc}{section}{ВВЕДЕНИЕ}

В ходе развития компьютерных систем объем данных стал достигать огромных размеров, поэтому многие операции стали выполняться за очень долгое время, так как обычно это простой перебор. Это вынудило создать новые алгоритмы, решающие проблему на порядок быстрее, чем стандартное решение пользовательского интерфейса. Это относится и к словарям, где одной из основных операций является операция поиска.

Целью данной лабораторной работы является приобретение навыка поиска по словарю при ограничении значения признака, заданном при
помощи лингвистической переменной.

Задачи, которые необходимо выполнить для достижения поставленной цели:
\begin{itemize}
	\item[---] формализовать объект и его признак;
	\item[---] составить анкету для заполнения респондентами;
	\item[---] провести анкетирование респондентов;
	\item[---] сформулировать 3–5 типовых вопроса на русском языке, целью которых является формирование запроса на поиск в словаре;
	\item[---] описать алгоритм поиска по словарю объектов, которые удовлетворяют ограничению, заданному в вопросе на ограниченном естественном языке;
	\item[---] описать структуру данных словаря, хранящего наименования объектов согласно варианту и числовое значение признака объекта;
	\item[---] реализовать описанный алгоритм поиска по словарю;
	\item[---] привести примеры запросов пользователя и выборки объектов из словаря, сформированной реализацией алгоритма поиска с использованием вопросов, составленными респондентами;
	\item[---] дать заключение о возможностях применения предложенного алгоритма и о его ограничениях.
\end{itemize}