\section{Технологическая часть}

В данном разделе будут рассмотрены средства разработки программного
обеспечения и детали реализации алгоритмов умножения матриц.

\subsection{Требования к программному обеспечению}
Программа принимает на входе вопрос, заданный пользователем на русском языке. Если вопрос не касается заданного объекта, то программа должна вывести сообщение о том, что на данный вопрос ответить невозможно, если же вопрос содержит объект, то программа должна вывести насколько возможно точный ответ.


\subsection{Средства реализации}
Для реализации программы был выбран язык Python~\cite{python}, так как для работы с русским языком не требуется установка сторонних библиотек.

\subsection{Реализация алгоритмов}

В листинге \ref{code:alg} приведена реализация алгоритма поиска в словаре.

\newpage
\begin{code}
	\captionsetup{justification=raggedright,singlelinecheck=off, margin=10pt}
	\captionof{listing}{Реализация алгоритма поиска в словаре}
	\label{code:alg}
	\inputminted
	[
	frame=single,
	framerule=0.5pt,
	framesep=20pt,
	fontsize=\footnotesize, 
	tabsize=4,
	linenos,
	numbersep=5pt,
	xleftmargin=10pt,
	]
	{text}
	{code/alg.py}
\end{code}
