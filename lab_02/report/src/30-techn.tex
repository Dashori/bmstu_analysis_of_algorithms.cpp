\chapter{Технологическая часть}

\section{Требования к ПО}

К программе предъявляется ряд требований:
\begin{itemize}
	\item входными данными являются две матрицы $A$ и $B$ (сначала вводятся их размерности, затем элементы);
	\item элементы матрицы -- целые числа;
	\item на выходе получается матрица $res$ -- результат умножения введенных пользователем матриц.
\end{itemize}

\section{Средства реализации}

В качестве языка программирования для реализации лабораторной работы был выбран C++ -- компилируемый, статически типизированный язык программирования общего назначения~\cite{cpp}. 

Данный выбор обусловлен поддержкой языком парадигмы объектно -- ориентированного программирования и наличием методов для замера процессорного времени.

Время работы реализованных алгоритмов было замерено с помощью библиотеки chrono~\cite{chrono}.

\section{Реализация алгоритмов}

В листингe \ref{lst:simple} приведена реализация стандартного алгоритма умножения матриц.
\lstinputlisting[caption={Реализация стандартного алгоритма умножения матриц}, label={lst:simple}]{code/simple.cpp}

В листингe \ref{lst:grape} приведена реализация алгоритма Винограда умножения матриц.
\lstinputlisting[caption={Реализация алгоритма Винограда умножения матриц}, label={lst:grape}]{code/grape.cpp}

В листингe \ref{lst:grapePro} приведена реализация оптимизированного алгоритма Винограда умножения матриц.
\lstinputlisting[caption={Реализация оптимизированного алгоритма Винограда умножения матриц}, label={lst:grapePro}]{code/grapePro.cpp}
