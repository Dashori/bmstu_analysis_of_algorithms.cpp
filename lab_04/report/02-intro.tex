\chapter*{Введение}
\addcontentsline{toc}{chapter}{Введение}


Одной из задач программирования является ускорение решения вычислительных задач. Один из способов ее решения -- использование параллельных вычислений.

В последовательном алгоритме решения какой-либо задачи есть операции, которые может выполнять только один процесс, например, операции ввода и вывода. Кроме того, в алгоритме могут быть операции, которые могут выполняться параллельно разными процессами. Способность центрального процессора или одного ядра в многоядерном процессоре одновременно выполнять несколько процессов или потоков, соответствующим образом поддерживаемых операционной системой, называют многопоточностью~\cite{multithreading}.

Для распараллеливания может быть рассмотрена задача поиска кратчайших путей между всеми парами вершин графа~\cite{diskra}. Данная задача решается при помощи алгоритма Флойда. Этот алгоритм был одновременно опубликован в статьях Роберта Флойда (Robert Floyd) и Стивена Уоршелла (Stephen Warshall) в 1962 году, хотя в 1959 году Бернард Рой (Bernard Roy) опубликовал практически такой же алгоритм, но это осталось незамеченным. Часто этот алгоритм называют алгоритмом Флойла-Уоршелла или алгоритмом Роя-Флойда. Далее будем использовать название алгоритм Флойда.

Цель лабораторной работы -- получить навык организации параллельных вычислений на базе нативных потоков.

Задачи лабораторной работы:
\begin{itemize}[label*=---]
	\item проанализировать последовательный и параллельный варианты алгоритма поиска кратчайших расстояний между всеми парами вершин графа;
	\item определить средства программной реализации выбранного алгоритма;
	\item реализовать разработанный алгоритм;
	\item провести сравнительный анализ по времени реализованного алгоритма;
	\item подготовить отчет о лабораторной работе.
\end{itemize}


