\chapter{Технологическая часть}

\section{Требования к ПО}

К программе предъявляется ряд требований:
\begin{itemize}
	\item входными данными являются количество вершин графа и матрица смежности;
	\item элементы матрицы смежности -- натуральные числа и число 0;
	\item на выходе -- результирующая матрица с кратчайшими расстояниями между всеми вершинами.
\end{itemize}

\section{Средства реализации}

В качестве языка программирования для реализации лабораторной работы был выбран C++ -- компилируемый, статически типизированный язык программирования общего назначения \cite{cpp}. 

Данный выбор обусловлен поддержкой языком парадигмы объектно -- ориентированного программирования, возможностью создавать нативные потоки и наличием методов для замера процессорного времени.

Время работы реализованных алгоритмов было замерено с помощью библиотеки chrono \cite{chrono}.

\section{Реализация алгоритмов}

В листингe \ref{lst:simple} приведена реализация последовательного алгоритма Флойда.
\lstinputlisting[caption={Реализация последовательного алгоритма Флойда}, label={lst:simple}]{code/simple.cpp}

В листингe \ref{lst:parallel} приведена реализация параллельного алгоритма Флойда.
\lstinputlisting[caption={Реализация параллельного алгоритма Флойда}, label={lst:parallel}]{code/parallel.cpp}

