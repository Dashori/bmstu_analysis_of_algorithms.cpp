\chapter*{Заключение}
\addcontentsline{toc}{chapter}{Заключение}

В ходе выполнения лабораторной работы поставленная цель была достигнута: был получен навык организации параллельных вычислений на базе нативных потоков.

В ходе выполнения лабораторной работы были решены все задачи:
\begin{itemize}[label*=---]
	\item проанализированы последовательный и параллельный варианты алгоритма поиска кратчайших расстояних между всеми парами вершин графа;
	\item реализованы выбранные алгоритмы;
	\item проведен анализ затрат работы реализованных алгоритмов по времени;
	\item на основе полученных в ходе экспериментов данных были сделаны выводы по поводу эффективности всех реализованных алгоритмов по времени;
	\item был подготовлен отчет по лабораторной работе.
\end{itemize}

Результат замерных экспериментов реализованных алгоритмов показал, что параллельная реализация алгоритма Флойда выигрывает последовательную реализацию, когда количество вершин графа больше 30. Если сравнивать по количеству потоков при фиксированном количестве вершин графа (и больше 30), то параллельная реализация работает быстрее при двух потоках в 1.35 раз и в 1.87 раз при восьми потоках.