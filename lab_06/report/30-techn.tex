\section{Технологическая часть}

\subsection{Требования к программному обеспечению}

К программе предъявляется ряд требований:
\begin{itemize}[label*=---]
	\item стоимости путей должны быть целыми числами;
	\item число городов должно быть больше 1;
	\item число дней должно быть больше 0;
	\item параметры муравьиного алгоритма должны быть вещественными неотрицательными числами;
	\item матрица должна задавать неориентированный граф;
	\item должно быть выдано сообщение об ошибке при некорректном вводе параметров.
\end{itemize}

\subsection{Средства реализации}

В качестве языка программирования для реализации лабораторной работы был выбран Python~\cite{python}. Данный выбор обусловлен наличием методов для замера процессорного времени.

Время работы реализованных алгоритмов было замерено с помощью библиотеки time~\cite{time}.

\subsection{Реализация алгоритмов}

В листингe \ref{lst:merge} приведена реализация алгоритма полного перебора.
\begin{code}
	\captionof{listing}{Реализация алгоритма полного перебора}
	\label{lst:merge}
	\inputminted
	[
	frame=single,
	framerule=0.5pt,
	framesep=20pt,
	fontsize=\small,
	tabsize=4,
	linenos,
	numbersep=5pt,
	xleftmargin=10pt,
	]
	{text}
	{code/pp.py}
\end{code}

В листингe \ref{lst:counting} приведена реализация муравьиного алгоритма.
\pagebreak
\begin{code}
	\captionof{listing}{Реализация муравьиного алгоритма}
	\label{lst:counting}
	\inputminted
	[
	frame=single,
	framerule=0.5pt,
	framesep=20pt,
	fontsize=\small,
	tabsize=4,
	linenos,
	numbersep=5pt,
	xleftmargin=10pt,
	]
	{text}
	{code/myr.py}
\end{code}

