\section*{ВВЕДЕНИЕ}
\addcontentsline{toc}{section}{ВВЕДЕНИЕ}

Задача коммивояжёра -- одна из самых известных задач комбинаторной оптимизации, заключающаяся в поиске самого выгодного маршрута, проходящего через указанные города хотя бы по одному разу с последующим возвратом в исходный город~\cite{kom}. Такая задача может быть решены при помощи полного перебора вариантов и эвристических алгоритмов. Алгоритмы, основанные на использовании эвристического метода, не всегда приводят к оптимальным решениям. Однако для их применения на практике достаточно, чтобы ошибка прогнозирования не превышала допустимого значения.

Целью данной лабораторной работы является сравнительный анализ метода полного перебора и эвристического метода на базе муравьиного алгоритма. 

Задачи лабораторной работы:
\begin{itemize}[label*=---]
	\item описать схемой алгоритма и реализовать метод полного перебора для решения задачи коммивояжёра.
	\item описать схемой алгоритма и реализовать метод решения задачи коммивояжёра на основе муравьиного алгоритма;
	\item выполнить оценку трудоёмкости описанных алгоритмов;
	\item провести сравнительный анализ двух рассмотренных методов решения задачи коммивояжёра;
	\item подготовить отчет по лабораторной работе.
\end{itemize}
