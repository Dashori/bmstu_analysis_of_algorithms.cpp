\section{Аналитическая часть}

\subsection{Алгоритм полного перебора}

Для решения задачи коммивояжёра алгоритм полного перебора предполагает рассмотрение всех возможных путей в графе и выбор наименьшего из них. Смысл перебора состоит в том, что перебираются все варианты объезда городов и выбирается оптимальный, что гарантирует точное решение задачи. Однако, при таком подходе количество возможных маршрутов очень быстро возрастает с ростом $n$ и сложность алгоритма равна $n!$.

\subsection{Муравьиный алгоритм}
Муравьиный алгоритм -- метод решения задач коммивояжёра на основании моделирования поведения колонии муравьев ~\cite{mur}. Каждый муравей определяет для себя маршрут, который необходимо пройти на основе феромона, который он получает во время прохождения, каждый муравей оставляет феромон на своем пути, чтобы остальные муравьи по нему ориентировались. В результате при прохождении каждым муравьем различного маршрута наибольшее число феромона остается на оптимальном пути.

Муравьи действуют согласно правилам:
\begin{itemize}[label*=---]
	\item муравей запоминает посещенные города, причем каждый город может быть посещен только один раз. Обозначим через $J_{i,k}$ список городов, которые посетил муравей $k$, находящийся в городе $i$;
	\item муравей обладает видимостью $\eta_{ij}$ - эвристическим желанием посетить город $j$, если муравей находится в городе i, причем
	\begin{equation}
		\label{d_func}
		\eta_{ij} = 1 / D_{ij},
	\end{equation}
	где $D_{ij}$ — стоимость пути из города $i$ в город $j$;
	\item муравей может улавливать след феромона - специального химического вещества. Число феромона на пути из города $i$ в город $j$ - $\tau_{ij}$.
\end{itemize}

Муравей выполняет следующую последовательность действий, пока не посетит все города:
\begin{itemize}[label*=---]
	\item выбирает следующий город назначения, основываясь на вероятностно-пропорциональном правиле \eqref{posib}, в котором учитываются видимость и число феромона:
	\begin{equation}
		\label{posib}
		P_{ij, k} = \begin{cases}
			\frac{\tau_{ij}^\alpha\eta_{ij}^\beta}{\sum_{l=1}^m \tau^\alpha_{il}\eta^\beta_{il}}, \textrm{если город j необходимо посетить;} \\
			0, \textrm{иначе,}
		\end{cases}
	\end{equation}
	где $\alpha$ - параметр влияния феромона, $\beta$ - параметр влияния видимости пути, $\tau_{ij}$ - число феромона на ребре $(ij)$, $\eta_{ij}$ - эвристическое желание посетить город $j$, если муравей находится в городе $i$. Выбор города является вероятностным, данное правило определяет ширину зоны города $j$, в общую зону всех городов $J_{i,k}$ бросается случайное число, которое и определяет выбор муравья;
	\item муравей проходит путь $(ij)$ и оставляет на нем феромон.
\end{itemize}

Информация о числе феромона на пути используется другими муравьями для выбора пути. Те муравьи, которые случайно выберут кратчайший путь, будут быстрее его проходить, и за несколько передвижений он будет более обогащен феромоном. Следующие муравьи будут предпочитать именно этот путь, продолжая обогащать его феромоном. 

После прохождения маршрутов всеми муравьями значение феромона на путях обновляется в соответствии со следующим правилом \eqref{update_phero_1}:

\begin{equation}
	\label{update_phero_1}
	\tau_{ij}(t+1) = (1-\rho)\tau_{ij}(t) + \Delta \tau_{ij},
\end{equation}
где $\rho$ - коэффициент испарения. Чтобы найденное локальное решение не было единственным, моделируется испарение феромона.

При этом
\begin{equation}
	\label{update_phero_2}
	\Delta \tau_{ij} = \sum_{k=1}^m \tau_{ij, k},
\end{equation}
где $m$ - число муравьев,
\begin{equation}
	\label{update_phero_3}
	\Delta\tau_{ij,k} = \begin{cases}
		Q/L_{k}, \textrm{если k-ый муравей прошел путь (i,j);} \\
		0, \textrm{иначе.}
	\end{cases}
\end{equation}
